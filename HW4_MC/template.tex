%%%%%%%%%%%%%%%%%%%%%%%%%%%%%%%%%%%%%%%%%
% Short Sectioned Assignment
% LaTeX Template
%%%%%%%%%%%%%%%%%%%%%%%%%%%%%%%%%%%%%%%%%

%----------------------------------------------------------------------------------------
%   PACKAGES AND OTHER DOCUMENT CONFIGURATIONS
%----------------------------------------------------------------------------------------

\documentclass[paper=a4, fontsize=11pt]{article} % A4 paper and 11pt font size

\usepackage[T1]{fontenc} % Use 8-bit encoding that has 256 glyphs
\usepackage{fourier} % Use the Adobe Utopia font for the document - comment this line to return to the LaTeX default
\usepackage[english]{babel} % English language/hyphenation
\usepackage{amsmath,amsfonts,amsthm} % Math packages

\usepackage{sectsty} % Allows customizing section commands
\allsectionsfont{\raggedright \normalfont\scshape} % Make all sections centered, the default font and small caps

\usepackage{fancyhdr} % Custom headers and footers
\pagestyle{fancyplain} % Makes all pages in the document conform to the custom headers and footers
\fancyhead{} % No page header - if you want one, create it in the same way as the footers below
\fancyfoot[L]{} % Empty left footer
\fancyfoot[C]{} % Empty center footer
\fancyfoot[R]{\thepage} % Page numbering for right footer
\renewcommand{\headrulewidth}{0pt} % Remove header underlines
\renewcommand{\footrulewidth}{0pt} % Remove footer underlines
\setlength{\headheight}{13.6pt} % Customize the height of the header

\usepackage{parskip}
\setlength{\parindent}{15pt}
\usepackage{graphicx}

\usepackage{subcaption}

\usepackage{booktabs}
\newcommand{\ra}[1]{\renewcommand{\arraystretch}{#1}}

%\numberwithin{equation}{section} % Number equations within sections (i.e. 1.1, 1.2, 2.1, 2.2 instead of 1, 2, 3, 4)
%\numberwithin{figure}{section} % Number figures within sections (i.e. 1.1, 1.2, 2.1, 2.2 instead of 1, 2, 3, 4)
%\numberwithin{table}{section} % Number tables within sections (i.e. 1.1, 1.2, 2.1, 2.2 instead of 1, 2, 3, 4)

\setlength\parindent{0pt} % Removes all indentation from paragraphs - comment this line for an assignment with lots of text

%----------------------------------------------------------------------------------------
%   TITLE SECTION
%----------------------------------------------------------------------------------------

\newcommand{\horrule}[1]{\rule{\linewidth}{#1}} % Create horizontal rule command with 1 argument of height

\title{ 
\normalfont \normalsize 
\textsc{Nuclear Engineering, UC Berkeley} \\ [25pt] % Your university, school and/or department name(s)
\horrule{0.5pt} \\[0.4cm] % Thin top horizontal rule
\huge HW4 for Numerical Simulation of Radiation Transport\\ % The assignment title
\horrule{2pt} \\[0.5cm] % Thick bottom horizontal rule
}

\author{Xin Wang} % Your name

\date{\normalsize\today} % Today's date or a custom date

\begin{document}
\clearpage\maketitle
\thispagestyle{empty}

\clearpage\tableofcontents
\thispagestyle{empty}
\pagebreak

\setcounter{page}{1}
%------------------------------------------------------------------------------------------------------

\begin{figure}
        \centering
        \begin{subfigure}[b]{0.3\textwidth}
                %\includegraphics[width=\textwidth]{gull}
                \caption{A gull}
                \label{fig:gull}
        \end{subfigure}%
        ~ %add desired spacing between images, e. g. ~, \quad, \qquad, \hfill etc.
          %(or a blank line to force the subfigure onto a new line)
        \begin{subfigure}[b]{0.3\textwidth}
                %\includegraphics[width=\textwidth]{tiger}
                \caption{A tiger}
                \label{fig:tiger}
        \end{subfigure}
        ~ %add desired spacing between images, e. g. ~, \quad, \qquad, \hfill etc.
          %(or a blank line to force the subfigure onto a new line)
        \begin{subfigure}[b]{0.3\textwidth}
                %\includegraphics[width=\textwidth]{mouse}
                \caption{A mouse}
                \label{fig:mouse}
        \end{subfigure}
        \caption{Pictures of animals}\label{fig:animals}
\end{figure}


% \begin{table*}\centering
% \ra{1.3}
% \begin{tabular}{@{}rrr@{}}\toprule
%     & isotope Z & weight fraction(\%) \\ \midrule
%     & 1         & 0.022100 \\
%     & 6         & 0.002484 \\
%     & 8         & 0.574930 \\
%     & 11        & 0.015208 \\
%     & 12        & 0.001266 \\
%     & 13        & 0.019953 \\
%     & 14        & 0.304627 \\
%     & 19        & 0.010045 \\
% \bottomrule
% \end{tabular}
% \caption{Ordinary concrete composition}
% \label{concrete_material_composition}
% \end{table*}

\end{document}