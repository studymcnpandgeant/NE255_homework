%%%%%%%%%%%%%%%%%%%%%%%%%%%%%%%%%%%%%%%%%
% Short Sectioned Assignment
% LaTeX Template
%%%%%%%%%%%%%%%%%%%%%%%%%%%%%%%%%%%%%%%%%

%----------------------------------------------------------------------------------------
%   PACKAGES AND OTHER DOCUMENT CONFIGURATIONS
%----------------------------------------------------------------------------------------

\documentclass[paper=a4, fontsize=11pt]{article} % A4 paper and 11pt font size

\usepackage[T1]{fontenc} % Use 8-bit encoding that has 256 glyphs
\usepackage{fourier} % Use the Adobe Utopia font for the document - comment this line to return to the LaTeX default
\usepackage[english]{babel} % English language/hyphenation
\usepackage{amsmath,amsfonts,amsthm} % Math packages

\usepackage{sectsty} % Allows customizing section commands
\allsectionsfont{\raggedright \normalfont\scshape} % Make all sections centered, the default font and small caps

\usepackage{fancyhdr} % Custom headers and footers
\pagestyle{fancyplain} % Makes all pages in the document conform to the custom headers and footers
\fancyhead{} % No page header - if you want one, create it in the same way as the footers below
\fancyfoot[L]{} % Empty left footer
\fancyfoot[C]{} % Empty center footer
\fancyfoot[R]{\thepage} % Page numbering for right footer
\renewcommand{\headrulewidth}{0pt} % Remove header underlines
\renewcommand{\footrulewidth}{0pt} % Remove footer underlines
\setlength{\headheight}{13.6pt} % Customize the height of the header

\usepackage{parskip}
\setlength{\parindent}{15pt}
\usepackage{graphicx}

\usepackage{booktabs}
\newcommand{\ra}[1]{\renewcommand{\arraystretch}{#1}}

%\numberwithin{equation}{section} % Number equations within sections (i.e. 1.1, 1.2, 2.1, 2.2 instead of 1, 2, 3, 4)
%\numberwithin{figure}{section} % Number figures within sections (i.e. 1.1, 1.2, 2.1, 2.2 instead of 1, 2, 3, 4)
%\numberwithin{table}{section} % Number tables within sections (i.e. 1.1, 1.2, 2.1, 2.2 instead of 1, 2, 3, 4)

\setlength\parindent{0pt} % Removes all indentation from paragraphs - comment this line for an assignment with lots of text

%----------------------------------------------------------------------------------------
%   TITLE SECTION
%----------------------------------------------------------------------------------------

\newcommand{\horrule}[1]{\rule{\linewidth}{#1}} % Create horizontal rule command with 1 argument of height

\title{ 
\normalfont \normalsize 
\textsc{Nuclear Engineering, UC Berkeley} \\ [25pt] % Your university, school and/or department name(s)
\horrule{0.5pt} \\[0.4cm] % Thin top horizontal rule
\huge HW4 for Numerical Simulation of Radiation Transport\\ % The assignment title
\horrule{2pt} \\[0.5cm] % Thick bottom horizontal rule
}

\author{Xin Wang} % Your name

\date{\normalsize\today} % Today's date or a custom date

\begin{document}

\maketitle % Print the title




\section*{Problem 1}
Calculate an integration $I = \int_{x_{min}}^{x_{max}}f(x) dx$ using rejection Monte Carlo method procedures:

1) Sample random points in the rectangle area [xmin, xmax]*[0, fmax]:

\begin{eqnarray}
x_p = x_{min} + (x_{max}-x_{min})*\xi_1 \nonumber\\
y_p = 0 + fmax * \xi_2
\end{eqnarray}

2) For each random point P (px, py), check if it's under the function curve, i.e. if $y_p \leq f(x_p)$.
If this is true, accept the point, otherwise reject the point. 

3) The probability for a point to be accepted is approximately 
\begin{eqnarray}
prob = \frac{N_{accept}}{N_{tot}}
\end{eqnarray}

4) The integral is I = Area * prob where Area = (xmax- xmin)*fmax

Assuming the true value for $\pi$ is 3.14159, the result for $\pi = 4\int_0^1 sqrt(1-x^2) dx$ and $\pi = 4\int_0^1 \frac{1}{1+x^2}$ with different numbers of samples is listed in table \ref{pb1}:

\begin{table*}[h]\centering
\ra{1.3}
\begin{tabular}{@{}cccc@{}}\toprule
  & nb & f1 relative error & f2 relative error \\ \midrule
    & 10         & 0.036     &0.108 \\
    & 100        & 0.082    &0.0196 \\
    & 1000         & 0.0186 & 0.0377 \\
    & 10000        & 0.0025 & 0.0013 \\
\bottomrule
\end{tabular}
\caption{Relative error of rejection Monte Carlo method for integral calculation}
\label{pb1}
\end{table*}


\section*{Problem 3}

\begin{table*}[h]\centering
\ra{1.3}
\begin{tabular}{@{}ccccc@{}}\toprule
 $N_{TOT}$ & $P_{escR}$ & $P_{escL}$ & $P_{absR}$ &$P_{absL}$ \\ \midrule
  125      & 10         & 0.036     &0.108 \\
  250      & 100        & 0.082    &0.0196 \\
  500      & 1000       & 0.0186 & 0.0377 \\
  1000     & 10000      & 0.0025 & 0.0013 \\
\bottomrule
\end{tabular}
\caption{Relative error of rejection Monte Carlo method for integral calculation}
\label{pb1}
\end{table*}

\section*{Problem 4}


a) The photon from the beam incident to the face of the slab, at z0=0 with the angle $\mu = 1$. 

The total collision cross section can be found online as attenuation coefficient:
\begin{eqnarray}
\rho_{concrete} = 2.3 (g/cm^3) \nonumber\\
\mu_{tot} = 4.557 * 0.01 * rho (cm)
\end{eqnarray}

Sample the distance to collision:
\begin{equation}
dist = -1/\mu_{tot} ln(\xi_3)
\end{equation}

b) So the first interaction would be at $z1 = z0 + dist = dist$ if z1 is within the slab [0, 5]cm.

c) Determine the collision type: absorption or scattering:
To do so, we need to at first calculate the scattering cross section. The mocroscopic total Compton scattering cross section can be obtain as:
\begin{eqnarray}
\sigma _{cs} &=& 2 \pi r_0 \{ \frac{1+\alpha} {\alpha ^2} [\frac{2(1+\alpha)}{1+2\alpha} - \frac{1}{\alpha} ln(1+2\alpha)] + [\frac{1}{2\alpha} ln(1+2\alpha) - \frac{1+3\alpha}{(1+2\alpha)^2}] \} [cm^2] \nonumber\\
\alpha &=& \frac{E_{in}}{m_0 c^2} \nonumber\\
r_0 &=& \frac{e^4}{(m_0 c^2)^2}
\end{eqnarray}

The macroscopic Compton scattering cross section is:
\begin{eqnarray}
\mu_{cs} = \sigma_{cs} * N
\end{eqnarray}

The atomic number density N can be calculated from :

\begin{eqnarray}
N = \rho * Av/ A
\end{eqnarray}

The average atomic mass A for ordinary concrete can be calculated from the material composition in table \ref{concrete_material_composition}

\begin{table*}\centering
\ra{1.3}
\begin{tabular}{@{}rrr@{}}\toprule
    & isotope Z & weight fraction(\%) \\ \midrule
    & 1         & 0.022100 \\
    & 6         & 0.002484 \\
    & 8         & 0.574930 \\
    & 11        & 0.015208 \\
    & 12        & 0.001266 \\
    & 13        & 0.019953 \\
    & 14        & 0.304627 \\
    & 19        & 0.010045 \\
\bottomrule
\end{tabular}
\caption{Ordinary concrete composition}
\label{concrete_material_composition}
\end{table*}

the Z/A ratio for ordinary concrete is 0.50932. 



\end{document}